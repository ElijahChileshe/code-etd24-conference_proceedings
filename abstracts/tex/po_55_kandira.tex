
    \begin{abstract_online}{Creating an Electronic Thesis Database (ETD) using OMEKA. The Zimbabwe Open University experience}{%
        \underline{Nobbie Kandira}$^{1}$, Edwin T Madziwo$^{1}$}{%
        }{%
        $^1$ Zimbabwe Open University, Zimbabwe}
        
        The purpose of this study was to establish the implementation experiences of the Zimbabwe Open University librarians in creating an Electronic Thesis and Dissertation Database (ETD) using OMEKA over Dspace software. The study explored the successes and challenges that were faced when coming up with the new institutional repository. The objectives of the study were to: (i) Establish the benefits of using OMEKA over DSpace. (ii) Establish the challenges faced in implementing the new ETD database. (iii) Highlight the benefits accrued from establishing the ETD. The study was qualitative in nature and used the phenomenology paradigm. Data was collected from all Librarians and Library IT staff at the Zimbabwe Open University that were involved in the implementation of the new repository. An interview guide was drafted in order to collect responses from the target population. The study was significant in that it gave a comparison of DSpace and OMEKA as two open source software that were available on the market. Results of the study revealed that lack of trained staff was one of the major problems in creating the ETDs.
    
    \end{abstract_online}
    

\begin{abstract_online}{Global Visibility of National ETD Repositories of G20 Countries: Comparative Studies with Respect to NDLTD’s Meta Repository}{%
    \underline{Sukanta Kumar Patra}$^{1}$}{%
    }{%
    $^1$Vidyasagar College for Women\newline{}
}

ETDs are primary information sources that originate from doctoral theses or dissertations submitted to universities for the doctoral award. The FAIR ETDs are those electronic theses and dissertations that are findable, accessible, interoperable, and reusable (FAIR). The study covers ETD initiatives by G20 member countries, comprising 19 countries (Argentina, Australia, Brazil, Canada, China, France, Germany, India, Indonesia, Italy, Japan, Republic of Korea, Mexico, Russia, Saudi Arabia, South Africa, Turkey, United Kingdom, and United States) and the European Union. The G20 member countries represent around 85% of the global GDP, over 75% of global trade, and about two-thirds of the world population. This study aims to analyze the current state of the ETD repositories of the G20 member countries and to describe their characteristics and performance in brief. The major objectives of the study are to analyze the importance of ETDs in a global context, to find out the linkage between Global ETD and ETDs of G20 member countries, and to provide a comparative scenario of ETD initiatives with respect to NDLTD repositories. Taking into consideration the findings, at the end of the study, there are some proposals and recommendations to further improve the status of national ETD repositories.

\end{abstract_online}


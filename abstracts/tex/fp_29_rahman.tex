
        \begin{abstract}{Designing a plan for sharing ETD among the University Libraries in Bangladesh}{%
            Zillur Rahman}{%
           Ahsanullah University Of Science and Technology, Bangladesh}{%
            }
 	 Purpose – ETDs have a great impact on exploring higher education in all higher educational institutions of the world. Sharing of ETDs among the universities can prevent repetition of research and guide new research direction.  It assures in averting plagiarism, controlling theft and copyright of one’s intellectual property. Hence, the purpose of this research is to design a plan for sharing ETD among the University Libraries in Bangladesh within the existing consortium and networking and resource sharing framework.
         
         Design/methodology/approach – This study uses both qualitative and quantitative approaches along with review of related literature. This paper begins with an overview of consortium and networking and resource sharing framework currently being employed by the university libraries of Bangladesh in electronic and digital platforms. It moved into a discussion of necessity of consortium and networking and resource sharing among the libraries as a whole and particularly in higher education. It closes by stressing a suitable model plan for developing countries like Bangladesh. Data has been collected through survey questionnaire.The conceptual and textual information related to the present study have been collected from primary and secondary sources of information such as books, journals, magazines, newspapers, conference proceedings, official documents, and unpublished sources. Websites of the sampled University Libraries have also been used for collecting information. Literature has been reviewed and extraction has been presented in the form of figure and tables. After processing and analysis of the data, appropriate physical meaning and interpretation to the numerical results in real life given for each of the Table and Figures.
      
         Findings – The study reveals that ETDs have a great role in disseminating knowledge among the academic communities. ETDs can be shared and impart in academic community locating any corner of the world. In such a paradox, no standard University Consortium has been built in Bangladesh since the independence in 1971. Several initiatives were taken by different organization including University Grants Commission (UGC), BANSDOC and universities individually but failed due to administrative skills and national guidelines.In case of university library consortium the situation is worse than other research organizations. There was no library consortium in Bangladesh before 2007. In 2006, an initiative was taken by the UGC of Bangladesh to form a Digital Resource Consortium for university libraries in Bangladesh for sharing integrated library resources including e-resources and computer database (Uddin \& Chowdhury, 2006, p. 490-3, Uddin, 2009, p. 196). In 2007, one consortium formed in Bangladesh was named Bangladesh INASP-PERI Consortium (BIPC), presently LiCoB with the participation of major public and private universities and a few research institutions under the guidance and supervision of Bangladesh Academy of Sciences (Uddin, 2009, p.196). Digital Archive on Agricultural Theses and Journal (DAATJ) found only established network in Bangladesh.  In 2009 UGC has formed UDL consortium under HEQEP to provide electronic resources among the universities with the financial support of World Bank.   
       
         Research limitations/implications – There are around 172 universities in Bangladesh now. Data collections from these universities are important to come in any concrete conclusion. The study depends on questionnaire on selected public and private universities only.  Practical implications – This study divulges a simple and cost effective Model Plan to implement within the existing framework of resource sharing to support in the digital environment. The research findings will direct the students of Library and Information Science, Faculty Members, and Policy Makers in further decision making and research. 
        \end{abstract}
        


        \begin{abstract}{Implementing Persistent Identifier Infrastructure for Effective Management of ETD Repositories: A Case Study from Chartered Institute of Personnel Management, Sri Lanka}{%
        Kamani Perera}{%
            Chartered Institute of Personnel Management, Sri Lanka}{%
            }

    Introduction: Electronic Theses and Dissertations (ETDs) have become integral components of scholarly communication, offering valuable insights into academic research and contributing to the advancement of knowledge across diverse fields. As the digital landscape continues to evolve, the management and accessibility of ETDs present both opportunities and challenges for academic institutions, libraries, and researchers alike. In this context, it is important to establish a vibrant infrastructure for persistent identifiers (PIDs) to proper management of ETDs. Persistent identifiers are essential tools that enable the unique and unambiguous identification of digital objects, ensuring their long-term discoverability, accessibility, and citability. In the context of ETD repositories, implementing a comprehensive PID infrastructure is paramount to streamline access, enhance interoperability, and facilitate scholarly communication. In this research study, it is described the significance of implementing a persistent identifier infrastructure for the effective management of ETD repositories. By examining the role of PIDs in improving discoverability, ensuring integrity, and fostering collaboration within the scholarly community, this study seeks to provide insights into best practices and strategies for implementing PID systems tailored to the unique needs of ETD repositories.

    Objectives
    1.To enhance discoverability
    2.To improve accessibility and long-term preservation
    3. To foster interoperability and integration among different ETD repositories
    4.To enable accurate and reliable citation of ETDs

    Methodology: A comprehensive review of existing literature, scholarly articles, reports, and best practices related to persistent identifiers, electronic theses and dissertations, repository management, and digital preservation were conducted. This review provides a foundational understanding of the current state-of-the-art, challenges, and opportunities in implementing persistent identifier infrastructure for ETD repositories. Various persistent identifier systems and frameworks such as DOI (Digital Object Identifier) and ARK (Archival Resource Key) were evaluated to identify the most suitable system for the repository.

    Results and Conclusion: Implementation of persistent identifier infrastructure led to a significant increase in the discoverability of ETDs within repository systems and external databases. By assigning unique identifiers to each document, ETDs became more easily searchable and identifiable, thereby improving their visibility among scholarly community. Further, it ensured the long-term accessibility and preservation of ETDs. Stable links provided by persistent identifiers enabled continued access to ETDs over time, even as technologies evolved and platforms changed. In this context, ETDs remained accessible to future generations. It promoted collaboration and knowledge sharing across institutional boundaries, facilitating interdisciplinary research and fostering a more interconnected scholarly ecosystem. Persistent identifiers streamlined the citation process for ETDs, promoting accuracy, reliability, and proper attribution to authors. Stable references provided by persistent identifiers were easily included in scholarly publications, citations, and bibliographies, supporting transparency, reproducibility, and scholarly communication.

    Conclusion: The implementation of persistent identifier infrastructure has proven to be instrumental in enhancing the management of ETD repositories. By providing unique PIDs for ETDs, this infrastructure has improved the discoverability, accessibility, interoperability, and citability of scholarly works, thereby facilitating broader dissemination of knowledge and fostering collaboration within the academic community. PIDs support the long-term preservation and accessibility of valuable research outputs for the benefit of future generations.

\end{abstract}


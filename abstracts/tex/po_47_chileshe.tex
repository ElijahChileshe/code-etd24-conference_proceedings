\begin{abstract_online}{Design and Implementation of an Interoperable Zambia National Electronic Thesis and Dissertation Portal}{%
    \underline{Elijah Chileshe}$^{1}$, Lighton Phiri $^{1}$}{%
    }{%
    $^1$ The University of Zambia}
    
    Introduction: Zambia boasts a significant number of Higher Learning Institutions (HEIs) contributing to a vibrant academic landscape marked by prolific scholarly research output. The Higher Education Authority (HEA) reports a total of 123 HEIs as of 2024 (Higher Education Authority, 2018). Many of these HEIs offer postgraduate programs that require students to produce thesis or dissertation manuscripts. Increasingly, such manuscripts are stored electronically, resulting in the proliferation of Electronic Theses and Dissertations (ETDs). Existing literature indicates that 10 HEIs have implemented Institutional Repositories (IRs), with eight still operational (Chisale & Phiri, 2023).

    Motivation: With the increase in the number of HEIs offering postgraduate programs, along with the rise in the adoption of IRs, the discoverability of these scholarly outputs is compromised. Additionally, assessing the quality of ETDs by entities such as the HEA is challenging. A potential solution to this problem is the implementation of a centralized platform for accessing ETDs.

    Methodology: A national ETD portal has been developed, primarily utilizing the Open Archives Initiative Protocol for Metadata Harvesting (OAI-PMH) to harvest ETD metadata from HEI IRs across Zambia. By leveraging OAI-PMH, the portal ensures efficient and standardized retrieval of metadata, enabling seamless integration of ETDs into its centralized repository. Ongoing monitoring and maintenance will be conducted to uphold the functionality and usability of the portal.

 	Results: It is anticipated that the centralized repository of ETDs will significantly improve accessibility to scholarly resources for researchers in Zambia. By providing a single platform for accessing ETDs from various institutions, researchers will save time and effort in locating relevant materials. Furthermore, the portal is expected to increase the visibility and recognition of research outputs from higher learning institutions in Zambia.

	References: Chisale, A., & Phiri, L. (2023, November 17). Towards Metadata Completeness in National ETD Portals for Improved Discoverability. 26th International Symposium on Electronic Theses and Dissertations. ETD 2023, Gujarat, India. [http://docs.ndltd.org/metadata/etd2023/9/index.html](http://docs.ndltd.org/metadata/etd2023/9/index.html)
    - Higher Education Authority. (2018, January 4). Ensuring Quality in Higher Education; Higher Education Authority. [https://hea.org.zm](https://hea.org.zm)

\end{abstract_online}


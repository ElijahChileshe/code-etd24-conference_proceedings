
    \begin{abstract_online}{Globalizing Knowledge: Leveraging Large Language Models to Enhance Accessibility of ETDs}{%
        \underline{Yinlin Chen}$^{1}$, William A Ingram$^{1,2}$, Edward A Fox$^{1,3}$}{%
        }{%
        $^1$ Virginia Tech, United States\newline{}$^2$ Networked Digital Library of Theses and Dissertations, United States}
        Electronic Theses and Dissertations (ETDs) encapsulate significant research findings and innovative ideas but often have limited visibility and accessibility, particularly in regions and disciplines with restricted digital reach. This workshop introduces an LLM-based application using a Retrieval-Augmented Generation (RAG) architectural approach to address these challenges. By utilizing LLMs to translate and standardize ETD metadata and content into a user’s native language, a unified vector database is established as a knowledge source for retrieving relevant information. This information is then supplied to the LLMs to generate comprehensive responses, enhancing searchability tailored to local or remote ETD collections.

        This approach improves the indexing and discoverability of ETDs and ensures accessibility across linguistic boundaries. During the workshop, we will present the details of this system's components, illustrating the program workflow and the interaction dynamics between the query, retrieval, and response generation phases. Participants will learn how to integrate these technologies into their digital library systems and repositories, adapting them to various institutional needs to enhance their ETD collections' global visibility and utility.
    
    \end{abstract_online}
    

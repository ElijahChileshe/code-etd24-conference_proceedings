\begin{abstract_online}{ETDs in Ensuring Quality Education for Economic Growth to Achieve Sustainable Development Goals (SDGs): Experience of SAARC Countries}{%
    \underline{Dr. Md. Zillur Rahman}$^{1}$}{%
    }{%
    $^1$ Ahsanullah University of Science and Technology *\newline{}
}

The purpose of this paper is to illustrate the trend of economic growth and the supply of skilled manpower, aiming to find a correlation between them and the role of Electronic Theses and Dissertations (ETDs) in producing human capital that will meet the demands of the Sustainable Development Goals (SDGs) among SAARC countries: Afghanistan, Bangladesh, Bhutan, India, Maldives, Nepal, Pakistan, and Sri Lanka. The paper also describes ETDs as an essential element of research and higher studies, highlighting their importance, nature, platforms, and initiatives in innovation to achieve the SDGs.

The main objectives of this paper are to find: 
1. The new role of libraries due to the UN's Agenda 2030 around the world.
2. The contribution of open content provided by UN Research4Life initiatives.
3. The role of ETDs and NDLTDs worldwide in accelerating economic growth in SAARC countries, ensuring quality education and skilled manpower.
4. Recommendations to overcome barriers in implementing these two goals.

Access to information is recognized as a priority in SDG 16 (target 16.10), ensuring public access to information and safeguarding fundamental freedoms in compliance with national legislation and international agreements. Additionally, culture (target 11.4), climate literacy (SDG 13), and ICT (targets 5b, 9c, and 17.8) are included in the SDGs. Universal literacy is also emphasized in the UN's 2030 Agenda. Obtaining a quality education (QE) is foundational for creating sustainable development (SD). Access to inclusive education can equip communities with tools to develop innovative solutions to the world’s greatest problems (United Nations, 2015). Without quality education, economic growth is unattainable. ETDs play a vital role in providing access to new scholarly knowledge, scientific publications, and valuable information to accelerate innovation in sustainable development. This research aims to reveal the impacts of ETDs on quality education and innovation in SD within SAARC countries, describing the SDGs and initiatives for creating a livable world for all.

**Methodology**: This paper is qualitative in nature. To highlight the importance of ETDs as a part of quality education in SAARC countries, data were collected from sources such as the United Nations, World Bank Databank, World Development Indicators, World Economic Forum, International Monetary Fund (IMF), and similar organizations. Additionally, national portals of the concerned countries, current literature, and personal initiatives were included.

**Expected Results**: Existing research in Library and Information Science is relevant to individual SDGs, serving as a link between them. The SDGs also focus on new sites for empirical research, inviting innovation in Library and Information Science's contribution to SD debates. ETDs are important tools for scholarly communication and innovation. They are becoming vital elements that significantly impact the creation of new knowledge, dissemination, and prevention of repetition in research. While the SAARC countries are growing economies, their education and human development are slower in comparison to developed nations, and the rate of new technology adaptation is also comparatively slow.

The results and recommendations contained in this paper should be of interest to authors, policymakers, funding agencies, and information professionals in both developing and developed countries (Chan & Costa, 2005).

**Conclusion**: Sustainable development depends on the success of the 17 SDGs, which are interdependent. Success in one goal does not equate to overall achievement. Therefore, ensuring quality education for sustainable economic growth is of utmost importance, and the role of ETDs is inevitable.

\end{abstract_online}



    \begin{abstract_online}{Optimizing Electronic Theses and Dissertations Management for Broad Audience Engagement at Dr. Magufuli Library of Mbeya University of Science and Technology, Tanzania.}{%
    \underline{Zainabu H LILUTU}$^{1}$}{%
}{%
    $^1$ Mbeya University of Science and Technology *\newline{}
}

Introduction: Dr. Magufuli Library is the keystone of academic and research activities at Mbeya University of Science and Technology (MUST), situated in Mbeya, Tanzania. Established with the mission of supporting the university’s educational, research, and outreach programs, the library serves as a vital resource center for students, faculty, and researchers. The management and accessibility of Electronic Theses and Dissertations (ETDs) at the library are vital for facilitating knowledge dissemination and ensuring broad audience engagement. Effective ETD systems not only support the academic community by preserving research outputs but also enhance the visibility and accessibility of scholarly work. Despite having an institutional repository, Dr. Magufuli Library faces significant challenges in managing and regularly updating the platform to keep it current. These challenges include inconsistent maintenance practices; additionally, there is a lack of clearly defined roles and responsibilities among the library staff regarding the management of the ETD repository. In addition, there are no updates about current resources and activities in the unit of the library from the university’s website. Information about the library and its activities seems to be from long ago. The study aims to explore challenges and ways to improve the handling of theses and dissertations for broader dissemination of research outputs, benefiting the academic community and extending the reach of the university’s scholarly contributions. Specifically, I intend to use Dr. Magufuli Library to optimize both the management efficiency of the repository/other digital platforms and the user experience, facilitating better utilization of the available resources.

Objectives:
1. To enhance ETD management efficiency at Mbeya University of Science and Technology.
2. To improve user accessibility and increase user engagement with available resources.
3. To promote broad dissemination of research.
4. To foster collaboration and feedback between library staff and researchers.
5. To integrate advanced technologies in Dr. Magufuli Library.

Methodological Approach: A mixed-methods approach is employed, including observations, interviews, and questionnaires, to gather comprehensive data on the current system. Observations are expected to reveal ETD management processes and user interactions, while semi-structured interviews with library staff, heads of faculties, and postgraduate students will highlight the system's strengths and weaknesses. Questionnaires will be distributed to postgraduate students and researchers to indicate difficulties in locating relevant ETDs and a need for enhanced accessibility, search capabilities, and user-friendly interfaces.

Anticipated Results: After the completion of the study, the findings are expected to suggest several ways to improve accessibility, management, and reach a broader audience. To implement digital platforms with advanced search functionalities and intuitive interfaces that can significantly enhance accessibility and user experience. To have a guideline that shows clear responsibility and oversight of defined roles and responsibilities among the library staff regarding the management of the ETD repository. In addition, assigning specific staff to deal with theses and dissertations in order to maintain accountability and responsibility. Moreover, the need for comprehensive training and support as well as targeted outreach including workshops and tutorials, to increase awareness and usage of ETD resources.

Conclusion: The study underscores the critical need for optimizing the ETD management system at Dr. Magufuli Library. Addressing the identified challenges through strategic improvements will not only streamline management processes but also enhance user engagement and accessibility for MUST and other researchers from outside the community, fostering a more effective dissemination of scholarly work.

\end{abstract_online}


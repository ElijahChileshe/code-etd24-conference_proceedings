\begin{abstract_online}{Enhancing Access to Scholarly Knowledge: Strategies for Promoting Open Access ETDs in Sri Lanka}{%
        \underline{Kamani Perera}$^{1}$, Ravindu Sachintha$^{1}$, Ashini Maxworth$^{2}$}{%
        }{
        $^1$ Chartered Institute of Personnel Management\newline{}
        $^2$ Sabaragamuwa University}
        {%
         }

    Introduction: With the emergence of digital technology and the advancement of open access initiatives, the accessibility and dissemination of scholarly knowledge have undergone a significant transformation. This has been evident in Sri Lanka, as in many other parts of the world. There is a pressing need to increase access to scholarly knowledge and promote the adoption of open access principles, particularly in the realm of ETDs. ETDs play a dynamic role as repositories of treasured research output, representing the high point of academic activities. Though there is a significant number of scholarly outputs produced by universities and research institutions in Sri Lanka, access to this knowledge remains limited. Traditional bottlenecks, such as subscription fees and restricted access models, prevent the broader dissemination of research findings, hindering the possible impact of Sri Lankan scholarship on a global scale. At this juncture, the elevation of open access ETD repositories is of paramount importance to Sri Lanka. Open access facilitates the unrestricted online availability of scholarly literature, enabling anyone, anywhere, to access, read, download, and distribute research outputs freely. 

    Objectives: • To Assess the Current Scenario in Sri Lanka • To Develop Strategies for Promoting Open Access ETDs • To Promote Collaboration and Capacity Building.
       
    Methodology: A comprehensive review of existing literature, scholarly articles, reports, and best practices related to open access ETDs was conducted. In-depth reviews were conducted on existing policies and guidelines related to ETDs, copyright, open access, and scholarly communication at the national and institutional levels in Sri Lanka. 

    Results: Key barriers to the accessibility and dissemination of ETDs in Sri Lanka, including limited awareness of open access principles, inadequate technical infrastructure, and legal and copyright concerns, were identified. It was revealed that there are institutions that have already implemented open access policies, while others need support and guidance to implement ETD repositories. Moreover, it is important to promote open access strategies such as policy development, repository infrastructure, metadata standards, and capacity building and collaboration among universities, libraries, government agencies, and international partners to support the implementation of open access ETD initiatives. Awareness programs and knowledge exchange are needed on a regular basis to build the capacity of researchers, librarians, and administrators in managing and promoting open access ETDs in both countries.
      
    Conclusion: Open access ETDs have the potential to significantly improve the accessibility and visibility of Sri Lankan scholarly knowledge, promoting collaboration, innovation, and knowledge exchange both domestically and internationally. Addressing bottlenecks such as technical infrastructure, legal concerns, and awareness gaps is crucial for the successful implementation of open access ETD initiatives. In the same vein, collaboration among stakeholders, including universities, libraries, government agencies, and international partners, is essential for advancing open access ETDs in Sri Lanka. Promoting a culture of open access is needed to guarantee the sustainability of open access ETD initiatives in Sri Lanka as a developing nation.
\end{abstract_online}


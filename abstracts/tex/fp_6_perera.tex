
        \begin{abstract}{Unlocking the Potential of ETDs: Implementation of a Novel ETD Repository in Chartered Institute of Personnel Management in Sri Lanka}{%
        Kamani Perera}{%
            Chartered Institute of Personnel Management, Sri Lanka}{%
            }

    Introduction:

    Electronic Theses and Dissertations (ETDs) represent a vital and widely utilized open-access resource for scholars and researchers globally. Within higher education, these digital versions of theses and dissertations serve as crucial reservoirs of information and knowledge for forthcoming research activities. ETDs cover both electronic versions and traditional hard copies of scholarly works submitted by researchers to their respective universities or institutions. In brief, an ETD can be defined as any thesis or dissertation mainly submitted, archived, and disseminated in electronic format. The creation of ETD repositories represents a revolutionary advancement in utilizing Information and Communication Technology (ICT) to organize institutional research materials systematically, nurturing a creative platform to drive forthcoming research activities. However, in Sri Lanka, despite the growing significance of ETDs, there exists a gap in the availability and accessibility of these resources, particularly in the field of Human Resource Management (HRM). The implementation of a novel ETD repository tailored specifically for HRM research in Sri Lanka is necessary.

    Objectives:

    To implement a novel ETD repository to serve the scholarly community to conduct novel research. To increase the accessibility and visibility of HRM research outputs by providing a centralized platform. To enhance the research culture and academic reputation of Sri Lanka by promoting open access to high-quality HRM scholarship through the ETD repository.

    Methods:

    Implementing a new ETD repository has become a crucial task, and initiatives have already been taken to upload the selected ETDs from bachelor and master level scholars. DSpace software is being used to build up the repository according to customary needs. Copyright clearance has duly been taken to host the institutional ETDs on the platform. Open access ETDs, which were collected through web navigation, are uploaded to the repository based on the institutional curriculum. Initially, ETDs are uploaded to the repository in descending order starting from 2023, the latest. Anti-plagiarism software is being used to check for plagiarism issues in ETDs before uploading to the repository. Metadata can be considered an integral part of the ETD lifecycle. Thus, it is ensured to adhere to internationally recognized metadata standards such as ETD-MS, established by the Networked Digital Library of Theses and Dissertations (NDLTD).

    Results and Conclusion:

    Quality ETDs are selected by the appointed academic committee. Moreover, proper guidance is provided to the newly enrolled students to encourage them to make use of the institutional repository and come up with innovative research titles for their academic assignments. Regular awareness programs are conducted to optimize the use of the ETD repository, and statistics are collected for annual evaluations; scholars can make comments for further improvement of the repository. This is the starting point of the scholarly journey by unlocking the potential of ETDs and inviting the new generation to conduct novel research using the ETD repository to find solutions for the burning questions in the field of HRM in the country. Moreover, this ETD repository links with Google Scholar and thereby generates Google Scholar Ranking for the research scholars who are the authors of the particular ETDs and Webometric Ranking for the hosting institute of the repository. It provides global visibility to HRM research. It preserves valuable resources and thereby reduces the dependence on physical copies and enables efficient search and retrieval of information. Further, ETDs contribute to environmental sustainability by decreasing paper consumption. In this context, ETD repositories create a dynamic and vibrant knowledge-sharing ecosystem that empowers the scholarly community to thrive in their careers and contribute meaningfully to building a vibrant research culture.

\end{abstract}

